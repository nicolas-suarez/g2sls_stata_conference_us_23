\documentclass[aspectratio=169]{beamer}
\setbeamertemplate{caption}[numbered]
\setbeamertemplate{footline}[frame number]
\setbeamertemplate{navigation symbols}{}

\usetheme{Stanford}

\usepackage[english]{babel}
\usepackage[utf8]{inputenc} 
\usepackage[T1]{fontenc}
\usepackage{graphicx}
\usepackage{booktabs}
\usepackage{adjustbox}
\usepackage{latexsym,amsmath,amssymb,amsfonts,bm} 
\usepackage{textcomp}
\usepackage{multirow}
\usepackage{pgffor}
\usepackage{tikz}
\usetikzlibrary{er,positioning}
\usepackage{tabu}
\usepackage{ragged2e}
\usepackage{appendixnumberbeamer}
\usefonttheme[stillsansserifsmall]{serif}
\usepackage{newcent}
\usepackage{color}
\usepackage{xcolor}
\usepackage{multicol}
\usepackage{verbatim}
\usepackage{tikz}
\usetikzlibrary{backgrounds,fit,trees,arrows,decorations.pathreplacing}
\usepackage{subcaption}
\usepackage[flushleft]{threeparttable}
\usepackage{array}
\usepackage[capposition=top]{floatrow}


\usepackage{stata}


\aboverulesep=0ex
\belowrulesep=0ex
\renewcommand{\arraystretch}{1.05}
%%%%%%%%%%%%%%%%%%%%%%%%%%%%%%%%%%%%%%%%%%%%%%
\title[Generalized 2SLS]{ \textbf{G2SLS: Generalized 2SLS procedure for Stata}}
\author[Nicolas Suarez (Stanford University)]{Nicolas Suarez-Chavarria}
\institute[Stanford University]
{\large{Stanford University}}
\date[]{\today}



\begin{document}
\frame{\titlepage}

\begin{frame}{Preview}
\begin{itemize}
\item I implement the generalized two-stage least squares procedure described in Bramoullé et al. (2009) to estimate peer effects models.
\pause
\item I extend their original framework to estimate peer effects models using OLS and to allow for independent variables without peer effects.
\pause
\item Short application to showcase the \texttt{2gsls} package.
\end{itemize}
\end{frame}


\begin{frame}{Outline}
  \tableofcontents
\end{frame}

\section{Motivation}

\begin{frame}{Motivation}
\begin{itemize}
\item If we want to estimate a linear-in-means regression, there are no readily available packages to do so.
\pause
\item Computing the mean outcomes and characteristics of peers with loops is hard and inefficient.
\pause 
\item To address this and the endogeneity problems in linear-in-means models, I developed the \texttt{2gsls} package.
\end{itemize}
\end{frame}


\section{Context}
\begin{frame}{Context}{Peer effects}
Peer effects can be classified into 3 categories:
\pause
\begin{itemize}
\item \textbf{Exogenous (or contextual) effects:} influence of exogenous peer characteristics on my outcomes.
\item \textbf{Endogenous effects:} influence of peer outcomes on my outcomes.
\item \textbf{Correlated effects:} individuals in the same reference group behave similarly because they face a common environment.
\end{itemize}
\end{frame}

\begin{frame}{Context}{Peer effects}
There are 2 main challenges when estimating a peer effects model:
\pause
\begin{enumerate}
\item It is difficult to distinguish real social effects (endogenous and exogenous) from correlated effects.
\item Reflection problem: Individuals simultaneously determine each other's outcomes. This endogeneity makes hard to distinguish between endogenous and exogenous effects.
\end{enumerate}
\pause
Generalized Two-Stage Least Squares tackles these 2 problems:
\pause
\begin{enumerate}
\item Adding network-level fixed effects controls for unobserved factors that affect individuals in the same group.
\item Using instrumental variables based on the network structure takes care of the endogeneity problem.
\end{enumerate}
\end{frame}

\begin{frame}{Context}{Econometric framework}
We start with a simple linear-in-means model:
\begin{equation}
y_i = \alpha + \textcolor<2>{red}{\beta} \frac{1}{n_i} \sum_{j \in P_i} y_j + \gamma x_i  + \textcolor<3>{red}{\delta}  \frac{1}{n_i} \sum_{j \in P_i} x_j + \varepsilon_i
\label{eq1}
\end{equation}
\begin{itemize}
\item $P_i$ are the peers of individual $i$.
\pause
\item $\textcolor<2>{red}{\beta}$ captures the endogenous peer effect.
\pause
\item $\textcolor<3>{red}{\delta}$ captures exogenous peer effects.
\end{itemize}

\end{frame}


\begin{frame}{Context}{Econometric framework}
We can rewrite this more generally using matrices:
\begin{equation}
y = \alpha \iota +  \textcolor<2>{red}{G} y \beta + X \gamma  +  \textcolor<2>{red}{G} X \delta + \varepsilon
\label{eq2}
\end{equation}
\pause
\begin{itemize}
\item $\textcolor<2>{red}{G}$ is an $N$-by-$N$ adjacency matrix representing the relationships between peers.
\pause
\item The $i$-th row of $G$ captures the relationship of individual $i$ with his peers. 
\end{itemize}
\end{frame}

\begin{frame}{Context}{Generalized Two-Stage Least Squares}
\begin{itemize}
\item Bramoullé et al. (2009) developed a procedure to estimate equation \eqref{eq2}.
\pause
\item We will rewrite our model as follows:
\begin{align*}
y & = \begin{bmatrix} \iota & Gy & X & GX \end{bmatrix} \begin{bmatrix} \alpha \\ \beta \\ \gamma \\ \delta  \end{bmatrix} + \varepsilon \\
\Leftrightarrow y & = \tilde{X} \theta + \varepsilon
\end{align*}
\item This model is identified if matrices $I$, $G$ and $G^2$ are linearly independent.
\end{itemize}
\end{frame}

\begin{frame}{Context}{Generalized Two-Stage Least Squares}
We follow these steps:
\begin{enumerate}
\item We define our instrument $ S = \begin{bmatrix} \iota & X & GX & G^2X \end{bmatrix} $ for $\tilde{X}$.
\pause
\item We estimate our model using 2SLS: 
$$\textcolor{red}{\hat{\theta}_{2SLS}}= (\tilde{X}' P \tilde{X})^{-1} \tilde{X}' P y  $$
with $P=S (S'S)^{-1} S'$.
\pause
\item We compute the predicted value of the outcome as:
$$ \textcolor{red}{\hat{y}_{2SLS}} = (I- \textcolor{red}{\hat{\beta}_{2SLS}} G)^{-1} \left(\textcolor{red}{\hat{\alpha}_{2SLS}}  + X \textcolor{red}{\hat{\gamma}_{2SLS}}  +  G X \textcolor{red}{\hat{\delta}_{2SLS}}  \right) $$
\end{enumerate}
\end{frame}


\begin{frame}{Context}{Generalized Two-Stage Least Squares}
\begin{itemize}
\item[4.] We build a new instrument for $\tilde{X}$:
$$\hat{Z}=\begin{bmatrix} \iota & G \ \textcolor{red}{\hat{y}_{2SLS}} & X & GX \end{bmatrix}$$
\pause 
\item[5.] We get our final estimator using standard IV:
$$ \textcolor{red}{\hat{\beta}_{G2SLS}} = (\hat{Z}'\tilde{X})^{-1} \hat{Z}'y$$
$$ \textcolor{red}{V\left(\hat{\beta}_{G2SLS}\right)} = (\hat{Z}'\tilde{X})^{-1} \hat{Z}'\ D \ \hat{Z} (\hat{Z}'\tilde{X})^{-1}$$
where $D$ is a diagonal matrix with the squared resids produced by $\hat{\beta}_{G2SLS}$.
\end{itemize}
\end{frame}

\begin{frame}{Context}{Variations to the model: Fixed effects} \label<2>{main}
\begin{itemize}
\item Bramoullé et al. (2009) also present a version of this model with network-specific unobservable factors:
\begin{equation}
y = \textcolor{red}{\sum_{l \in G} \alpha_l} +  {G} y \beta + X \gamma  + {G} X \delta + \varepsilon
\label{eq3}
\end{equation}
where $\alpha_l$ is common to all individuals in the $l$-th component of the network. 
\pause 
\item We can transform this model by multiplying it by $(I-G)$ to get rid of these unobservable effects. \hyperlink{fe_details}{\beamergotobutton{G2SLS with FE details}}
\end{itemize}
\end{frame}

\begin{frame}{Context}{Variations to the model: Direct effects}
\begin{itemize}
\item I extended the previous framework to allow for independent variables without peer effects:

$$y = \alpha +  {G} y \beta + X_1 \gamma  + {G} X_1 \delta  + \textcolor<2>{red}{X_2 \psi} + \varepsilon$$

\pause
\item $\textcolor{red}{\psi}$ captures the effects of our direct variables $\textcolor{red}{X_2}$.
\end{itemize}
\end{frame}

\section{Implementation}
\begin{frame}{Implementation}{G2SLS syntax}
  \scalebox{1}{\begin{minipage}{\hsize}
      \begin{stsyntax}
        g2sls
        \depvar\
        {\it indepvars}\
        \optif\
        \optin\
        ,
        \underbar{adj}acency(Mata matrix)
        \optional{\underbar{row}
          \underbar{fixed}
          \underbar{ols}
          \underbar{dir}ectvariables(\varlist)
          \underbar{level}(\num)}    
      \end{stsyntax}
  \end{minipage}}\\
  \vspace{0.5cm}
  \pause
  \textbf{Options}:
  \begin{itemize}
  \item \texttt{adjacency}: Mata matrix containing an $N$ by $N$ matrix of adjancency.
  \item \texttt{row}: row normalizes the adjacency matrix, so each row sums 1.
  \item \texttt{fixed}: adds component-level fixed effects.
  \item \texttt{ols}: reports OLS results instead of IV.
  \item \texttt{directvariables}: independent variables that will not have an exogenous effect.
  \item \texttt{level}: set confidence level for reported confidence intervals.
  \end{itemize}
\end{frame}

\section{Application}
\begin{frame}{Application}{Context}
\begin{itemize}
\item Peer effects for college students in Chile between 2012 and 2019.
\pause
\item 8 cohorts of approximately 500 students each from the Business and Economics school of the University of Chile.
\pause 
\item Students are randomly assigned to their first semester classes. We define their peers as the students they share at least 1 class with.
\pause 
\item Our adjacency matrix will be block diagonal, with each cohort being represented by a block.
\end{itemize}

\end{frame}


\begin{frame}{Application}{Data}
\adjustbox{max width= 0.85\linewidth, center,trim=0 0 1.2cm 0, clip=true}{
    \begin{minipage}{\hsize}
    \begin{stlog}
      . describe gpa_first adm_score aff_action female major*
{\smallskip}
Variable      Storage   Display    Value
    name         type    format    label      Variable label
\HLI{176}
gpa_first       float   \%9.0g                 First semester GPA
adm_score       float   \%9.0g                 Admission score
aff_action      byte    \%9.0g                 Affirmative action
female          byte    \%9.0g                 Female
major_econ      float   \%9.0g                 Major in Economics
major_buss      float   \%9.0g                 Major in Business
{\smallskip}
. list gpa_first adm_score aff_action female major* in 1/5
{\smallskip}
     {\TLC}\HLI{65}{\TRC}
     {\VBAR} gpa_first   adm_score   aff_ac{\tytilde}n   female   major_{\tytilde}n   major_{\tytilde}s {\VBAR}
     {\LFTT}\HLI{65}{\RGTT}
  1. {\VBAR}  .1698871   -1.262415          0        1          0          0 {\VBAR}
  2. {\VBAR}  .7442471      .44189          0        0          1          0 {\VBAR}
  3. {\VBAR} -2.991099    .4029151          0        0          0          0 {\VBAR}
  4. {\VBAR}  .4959475    2.504061          0        0          1          0 {\VBAR}
  5. {\VBAR}  .7618809    2.822953          0        0          1          0 {\VBAR}
     {\BLC}\HLI{65}{\BRC}
 
    \end{stlog}
    \end{minipage}}
\end{frame}

\begin{frame}{Application}{Standard IV model}
\adjustbox{max width= \linewidth, center}{
    \begin{minipage}{\hsize}
    \begin{stlog}
      . g2sls gpa_first female aff_action adm_score, row adj(G)
{\smallskip}
                                                  Number of obs =       4308
\HLI{13}{\TOPT}\HLI{64}
   gpa_first {\VBAR} Coefficient  Std. err.      t    P>|t|     [95\% conf. interval]
\HLI{13}{\PLUS}\HLI{64}
       _cons {\VBAR}   .0111257   .0778849     0.14   0.886    -.1415689    .1638204
 gpa_first_p {\VBAR}   .5676393   .4738957     1.20   0.231    -.3614408    1.496719
      female {\VBAR}   .1856059   .0177705    10.44   0.000     .1507666    .2204452
  aff_action {\VBAR}   .0935423   .0413983     2.26   0.024     .0123802    .1747044
   adm_score {\VBAR}   .3069133   .0187414    16.38   0.000     .2701704    .3436562
    female_p {\VBAR}  -.2034284   .1893837    -1.07   0.283    -.5747183    .1678614
aff_action_p {\VBAR}  -.0598223   .1047165    -0.57   0.568    -.2651206    .1454761
 adm_score_p {\VBAR}  -.3068539   .0777929    -3.94   0.000    -.4593681   -.1543397
\HLI{13}{\BOTT}\HLI{64}
 
    \end{stlog}
    \end{minipage}}
\end{frame}

\begin{frame}{Application}{IV model with fixed effects}
\adjustbox{max width= \linewidth, center}{
    \begin{minipage}{\hsize}
    \begin{stlog}
      . g2sls gpa_first female aff_action adm_score, row adj(G) fixed
{\smallskip}
                                                  Number of obs =       4308
Controlling for component-level fixed effects
\HLI{13}{\TOPT}\HLI{64}
   gpa_first {\VBAR} Coefficient  Std. err.      t    P>|t|     [95\% conf. interval]
\HLI{13}{\PLUS}\HLI{64}
 gpa_first_p {\VBAR}   .0066238    1.45047     0.00   0.996    -2.837045    2.850293
      female {\VBAR}   .1870434   .0183427    10.20   0.000     .1510823    .2230045
  aff_action {\VBAR}   .0887381   .0427942     2.07   0.038     .0048394    .1726368
   adm_score {\VBAR}   .3074021   .0190128    16.17   0.000     .2701271    .3446771
    female_p {\VBAR}   .0508922    .427888     0.12   0.905    -.7879889    .8897733
aff_action_p {\VBAR}   .0147204   .2325366     0.06   0.950    -.4411712     .470612
 adm_score_p {\VBAR}  -.1949845   .2778821    -0.70   0.483    -.7397767    .3498076
\HLI{13}{\BOTT}\HLI{64}
 
    \end{stlog}
    \end{minipage}}
\end{frame}

\begin{frame}{Application}{OLS model}
\adjustbox{max width=\linewidth, center}{
    \begin{minipage}{\hsize}
    \begin{stlog}
      . g2sls gpa_first female aff_action adm_score, row adj(G) ols
{\smallskip}
                                                  Number of obs =       4308
\HLI{13}{\TOPT}\HLI{64}
   gpa_first {\VBAR} Coefficient  Std. err.      t    P>|t|     [95\% conf. interval]
\HLI{13}{\PLUS}\HLI{64}
       _cons {\VBAR}  -.0233233   .0768411    -0.30   0.762    -.1739714    .1273248
 gpa_first_p {\VBAR}  -.7923793   .1912952    -4.14   0.000    -1.167417   -.4173421
      female {\VBAR}   .1847655   .0183224    10.08   0.000     .1488442    .2206869
  aff_action {\VBAR}   .1040999   .0424841     2.45   0.014     .0208091    .1873906
   adm_score {\VBAR}   .3188386    .016474    19.35   0.000      .286541    .3511363
    female_p {\VBAR}  -.0519749   .1807673    -0.29   0.774     -.406372    .3024222
aff_action_p {\VBAR}   .0464968   .0974559     0.48   0.633     -.144567    .2375605
 adm_score_p {\VBAR}   -.106855    .043616    -2.45   0.014    -.1923649   -.0213452
\HLI{13}{\BOTT}\HLI{64}
 
    \end{stlog}
    \end{minipage}}
\end{frame}

\begin{frame}{Application}{IV model with direct effects}
\adjustbox{max width= \linewidth, center,}{
    \begin{minipage}{\hsize}
    \begin{stlog}
      . g2sls gpa_first female aff_action adm_score, row adj(G) directvariables(major_*)
{\smallskip}
                                                  Number of obs =       4308
\HLI{13}{\TOPT}\HLI{64}
   gpa_first {\VBAR} Coefficient  Std. err.      t    P>|t|     [95\% conf. interval]
\HLI{13}{\PLUS}\HLI{64}
       _cons {\VBAR}  -.5948202   .0914547    -6.50   0.000    -.7741186   -.4155219
 gpa_first_p {\VBAR}   -4.26719   .5560077    -7.67   0.000    -5.357252   -3.177128
      female {\VBAR}   .1861742   .0170568    10.91   0.000     .1527341    .2196143
  aff_action {\VBAR}   .0755271   .0389122     1.94   0.052    -.0007609    .1518151
   adm_score {\VBAR}   .2892999    .018163    15.93   0.000     .2536911    .3249087
    female_p {\VBAR}   .8655189   .2057867     4.21   0.000     .4620708    1.268967
aff_action_p {\VBAR}  -.2825957    .096932    -2.92   0.004    -.4726325   -.0925588
 adm_score_p {\VBAR}   .1189513   .0754862     1.58   0.115    -.0290407    .2669433
  major_econ {\VBAR}   .6840927   .0416389    16.43   0.000     .6024591    .7657264
  major_buss {\VBAR}   .5122815   .0397691    12.88   0.000     .4343135    .5902495
\HLI{13}{\BOTT}\HLI{64}
 
    \end{stlog}
    \end{minipage}}
\end{frame}

\begin{frame}{Application}{Presenting results}
We can use \textbf{\texttt{estimates store}} and \textbf{\texttt{estout}} to organize our results:\\
\vspace{0.37cm}
\centering
\adjustbox{max width= 0.8\linewidth, center}{
  \begin{tabular}{c|ccc|ccc}
\toprule
\textbf{Variable} & \multicolumn{3}{|c|}{\textbf{OLS}} & \multicolumn{3}{|c}{\textbf{G2SLS}} \\
\midrule
\csname @@input\endcsname "results.tex"
\midrule
\textbf{Cohort level fixed effects} & No &  Yes &  Yes &  No &  Yes &  Yes  \\
\bottomrule
\end{tabular}%
}
\end{frame}


\section{Concluding remarks}



\begin{frame}
  \begin{center}
  \Huge \textbf{Thank you!}
  \end{center}


  nsuarez@stanford.edu
\end{frame}



\appendix


\begin{frame}{References}
\begin{itemize}
\item Bramoullé, Y., Djebbari, H., \& Fortin, B. (2009). Identification of peer effects through social networks. Journal of econometrics, 150(1), 41-55.
\end{itemize}
\end{frame}

\begin{frame}{Generalized Two-Stage Least Squares}{Model with fixed effects} \label{fe_details}
\begin{itemize}
\item We start by pre-multiplying equation \eqref{eq3} by $(I-G)$:
$$(I-G) y = (I-G) {G} y \beta + (I-G) X \gamma  + (I-G) {G} X \delta + \varepsilon$$
\item We will rewrite our model as follows:
\begin{align*}
(I-G) y & = \begin{bmatrix} (I-G) Gy & (I-G) X & (I-G) GX \end{bmatrix} \begin{bmatrix} \beta \\ \gamma \\ \delta  \end{bmatrix} + \varepsilon \\
\Leftrightarrow (I-G) y & = \tilde{X} \theta + \varepsilon
\end{align*}
\item This model is identified if matrices $I$, $G$, $G^2$ and $G^3$ are linearly independent.
\end{itemize}
\end{frame}

\begin{frame}{Generalized Two-Stage Least Squares}{Model with fixed effects}
We follow these steps:
\begin{enumerate}
\item We define our instrument $ S = \begin{bmatrix} (I-G) X & (I-G) GX & (I-G) G^2X \end{bmatrix} $ for $\tilde{X}$.
\item We estimate our model using 2SLS: 
$$\textcolor{red}{\hat{\theta}_{2SLS}}= (\tilde{X}' P \tilde{X})^{-1} \tilde{X}' P (I-G) y  $$
with $P=S (S'S)^{-1} S'$.
\item We compute the predicted value of the outcome as:
$$ \textcolor{red}{\hat{y}_{2SLS}} = (I-G)^{-1}(I- \textcolor{red}{\hat{\beta}_{2SLS}} G)^{-1} (I-G) \left(\ X \textcolor{red}{\hat{\gamma}_{2SLS}}  +  G X \textcolor{red}{\hat{\delta}_{2SLS}}  \right) $$
\end{enumerate}
\end{frame}


\begin{frame}{Generalized Two-Stage Least Squares}{Model with fixed effects}
\begin{itemize}
\item[4.] We build a new instrument for $\tilde{X}$:
$$\hat{Z}=\begin{bmatrix} (I-G) G \ \textcolor{red}{\hat{y}_{2SLS}} & (I-G)X & (I-G)GX \end{bmatrix}$$
\item[5.] We get our final estimator using standard IV:
$$ \textcolor{red}{\hat{\beta}_{G2SLS}} = (\hat{Z}'\tilde{X})^{-1} \hat{Z}'(I-G)y$$
$$ \textcolor{red}{V\left(\hat{\beta}_{G2SLS}\right)} = (\hat{Z}'\tilde{X})^{-1} \hat{Z}'\ D \ \hat{Z} (\hat{Z}'\tilde{X})^{-1}$$
where $D$ is a diagonal matrix with the squared resids produced by $\hat{\beta}_{G2SLS}$.
\end{itemize}
\hyperlink{main}{\beamerreturnbutton{back}}
\end{frame}


\end{document}

